Die Clients sind daf�r zust�ndig um dem Benutzer ein Interface zu bieten, mit
dem er den Webservice nutzen kann. In Zuge dieser Arbeit werden nur graphische
Benutzerinterfaces betrachtet, da diese f�r die Benutzer einfacher zu erlernen
sind und auch einen einfachen Workflow bieten. Umgesetzt wurden der native
Windows-Client und ein Web-Client, der lesenden Zugriff erm�glicht. Beschrieben
wird der gemeinsame Teil, der die Daten versendet und empf�ngt.
\subsection{Anforderungen}
Jeder der Clients muss eine Komponente besitzen 
Die Daten werden vom Backend �ber einen HTTP-Client in JSON empfangen und dem
Benutzer in einer grafischen Oberfl�che entsprechend pr�sentiert. Dabei werden
die Daten als Objekte in der Anwendung gehalten. Um ausfallsicher zu arbeiten
wird jede �nderung der Daten direkt an das Backend gesendet.

\subsection{Implementierung}
	F�r den nativen Client wird die Programmiersprache C\# so wie das Framework
	einer Windows Forms Anwendung verwendet. Wie in Kapitel \ref{Windows Forms}
	beschrieben verwendet eine solche Anwendung das .Net Framework. Dies erm�glicht 
	
	\subsubsection{Model}

\subsubsection{WEB-Request}

\paragraph{Serialize JSON}
