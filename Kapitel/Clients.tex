Die Clients sind daf�r zust�ndig um dem Benutzer ein Interface zu bieten, mit
dem er den Webservice nutzen kann. In Zuge dieser Arbeit werden nur graphische
Benutzerinterfaces betrachtet, da diese f�r die Benutzer einfacher zu erlernen
sind und auch einen einfachen Workflow bieten. Umgesetzt wurden der native
Windows-Client und ein Web-Client, der lesenden Zugriff erm�glicht. Beschrieben
wird der gemeinsame Teil, der die Daten versendet und empf�ngt. Die eigentlichen
Benutzeroberfl�chen werden dann in respektiven Kapiteln behandelt.
\subsection{Anforderungen}
Jeder der Clients muss eine Komponente besitzen, um Daten ans Backend zu senden
und zu erhalten. Dies wird im Falle der .NET-Basierten Clients �ber die
HttpWebRequest-Klasse aus der System.Net-Bibliothek. Mithilfe dieser Methode
wird der Zugriff auf das Backend abstrahiert, �hnlich einer DAO-Schicht, der
TRManager\_http\_client.
Die Funktionsweise dieser Klasse wird in Kapitel \ref{sssec:http_client} auf
Seite \pageref{sssec:http_client} n�her erl�utert.\\
Neben der Komponente zum Daten empfangen sollen die Clients es dem Anwender
erm�glichen, Vorf�lle einzutragen und wieder auszutragen (entlassen). F�r den
administrativen Benutzer soll zudem die M�glichkeit bestehen, Datens�tze zu
verwalten (Im Sinne von bearbeiten, neue erstellen und l�schen) und Daten zu
im- und exportieren. 

\subsection{Implementierung}
	Jeder im Zuge dieser Arbeit implementierte Client wurde in C\# mit dem
	.NET-Framework geschrieben, der Web-Client verwendet zus�tzlich f�r die
	Web-Funktionalit�t das ASP.NET-Framework. Dies bedeutet, dass ein gro�er Teil
	des Codes wiederverwendet werden kann, wie etwa die Datenmodell-Klassen und die
	Methoden zum empfangen und senden von Daten �ber das Netzwerk (respektive �ber
	HTTP). Die gemeinsame Basis der Clients wird in diesem Kapitel beschrieben.
	\subsubsection{Model}
		Das Model enth�lt die Datenstrukturen, in der das Backend "`spricht"'. Diese
		werden vom Backend festgelegt und m�ssen in den Clients exakt umgesetzt
		werden. Zu den Model-Klassen geh�ren wie in Kapitel \ref{sssec:model} auf
		Seite \pageref{sssec:model} beschrieben:
		\begin{itemize}
		  \item Teacher
		  \item Form (Schulklasse)
		  \item Student (Sch�ler)
		  \item Incident (Vorfall)
		  \item Comment (Vorgefertigter Kommentar)
		\end{itemize}
		Der einzige Unterschied zum serverseitigen Modell besteht darin, dass f�r die
		Basis-Klasse "`LazyObject"' (Siehe Anhang) keine Klasse implementiert wurde,
		da die Funktionalit�t, die LazyObject erf�llt (Flag zum Anzeigen ob das
		Objekt vollst�ndig geladen ist und Hilfsklassen), nur auf dem Server ben�tigt
		wird und f�r den Client bis auf die vererbte ID uninterresant ist. Daraus
		ergibt sich auch die �nderung, dass die Klassen im Gegensatz zum Model im
		Backend die ID als direktes Attribut f�hren.\\
		\lstinputlisting[label=teacher_client, caption=Teacher-Model-Klasse im
		Client]{listings/Teacher.cs}
		Listing \ref{teacher_client} auf Seite \pageref{teacher_client} zeigt als
		Beispiel einen Auszug aus der Teacher-Model-Klasse des Clients. Was hierbei zu
		beachten ist, dass die Attribute exakt so benamt sind wie im Backend. Dies ist
		sp�ter f�r die Serialisierung und Deserialisierung der Daten mit JSON wichtig,
		da sonst die Attribute nicht richtig zugewiesen werden k�nnen. Zudem wird mit
		"`\[JsonConstructor\]"' der Konstruktor markiert, mit dem die Objekte
		deserialisiert werden sollen. Die Mechanik, die mit JSON umgeht wird in
		Kapitel \ref{ssssec:s_json} auf Seite \pageref{ssssec:s_json} erl�utert.\\
	\subsubsection{HTTP-Client}\label{sssec:http_client}
		Der HTTP-Client ist sozusagen das "`Herst�ck"' der Clients. Er empf�ngt die
		Daten und bereitet diese f�r die Verwendung auf und serialisiert und sendet
		Objekte an das Backend. Es abstrahiert �hnlich einer DAO-Schicht die
		"`Abfragen"' (in diesem Kontext Web-Requests) vom Programmierer und bietet
		eine komfortable M�glichkeit, den Web-Service zu nutzen. Der HTTP-Client
		benutzt Generics und l�sst sich so an fast jeden Webservice anpassen.\\
		Es wurden f�r die wichtigsten CRUD-Methoden Methoden implementiert:
		\begin{itemize}
		  \item add (POST)
		  \item addBulk (POST)
		  \item edit (PUT)
		  \item delete (DELETE)
		  \item getByID (GET)
		  \item getAll (GET)
		\end{itemize}
		\paragraph{Serialize JSON}\label{ssssec:s_json}
		
\cleardoublepage
