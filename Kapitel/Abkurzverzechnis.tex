\begin{acronym}
	\acro{ER}{Entity Relationship}
	\acro{REST}{Representational State Tranfer}
	\acro{HTTP}{Hyper Text Transfer Protocol}
	\acro{HTTPS}{Hyper Text Transfer Protocol Secure}
	\acro{LAN}{Local Area Network}
	\acro{AJAX}{Asynchronous Javascript and XML}
	\acro{ASP}{Active Server Pages}
	\acro{JSF}{Java Server Faces}
	\acro{ORM}{Object Relational Mapping}
	\acro{DRY}{Don't repeat yourself}
	\acro{JDBC}{Java Database Connectivity}
	\acro{JPA}{Java Persistence API}
	\acro{JMS}{Java Message Service}
	\acro{ASP}{Active Server Pages}
	\acro{MSIL}{Microsoft Intermediate Language}
	\acro{IoC}{Inversion of Control}
	\acro{AOP}{Aspect Oriented Programming}
	\acro{WWW}{World Wide Web}
	\acro{VPN}{Virtual private Network}
	\acro{OSS}{Open Source Software}
	\acro{DBMS}{Datenbankmanagement-System}
	\acro{URI}{Uniform Resource Identifier}
	\acro{MVC}{Model View Controller}
\end{acronym}
%\ac{KDE}   %Gibt bei der ersten Verwendung die Langform mit der Abk�rzung in Klammern aus, als dann stets die Kurzform.
%\acs{KDE}  %Gibt die Abk�rzung aus.
%\acf{KDE}  %Gibt die Langform und die Kurzform aus.
%\acl{KDE}  %Gibt nur die Langform ohne die Kurzform aus.
