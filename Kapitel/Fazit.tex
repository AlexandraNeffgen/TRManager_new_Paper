Im Laufe des Projektes wurden die User-Stories durchgearbeitet. Dabei wurden
einige umgesetzt, andere neu bewertet und weitere auch g�nzlich gestrichen.
Folgend werden die wichtigsten User-Stories und deren Bearbeitung dargestellt.

Die wesentlichen Grundfunktionen mit der Priorit�t eins wurden im Rahmen dieses
Projektes vollst�ndig umgesetzt. Die Anforderungen zu Priorit�t Zwei wurden im
wesentlichen mit kleineren �nderungen ebenfalls umgesetzt. Die druckfertige
Nachricht nach dem dritten bzw. den f�nften Trainingsraum-Besuch ist bisher noch
nicht umgesetzt.
Die Anforderungen mit der Priorit�t Drei bis F�nf wurden noch nicht umgesetzt
und m�ssen noch gepr�ft werden. Zu diesen Anforderungen geh�rt im
wesentlichen die Verwendung von Mailadressen. Um diese Funktionen umzusetzen
m�ssen mehrere Bedingungen erf�llt sein. Zum einem werden die Mailadressen der
Lehrer ben�tigt und zum anderen muss ein Server zur Verf�gung gestellt werden,
welcher die Mails an die Lehrer verschicken kann. Das Verwaltungstool
SchILD-NRW, aus welchem s�mtliche Daten exportiert werden speichert nicht die
Mailadressen der Lehrer. Au�erdem steht dem Berufskolleg mit dem kooperiert
wird weder die Mittel noch die Kompetenzen f�r einen eigenen Mail-Server zur
Verf�gung. \\
Die Statistiken und das Archiv wurden ebenfalls noch nicht umgesetzt. Die
Vorbereitungen, wie etwa der Export der Daten wurden sind allerdings schon
implementiert.\\
Die unterst�tzende Funktion des externen Zugriffes wurde mittels des Web-Clients
realisiert. Weiteres dazu in Kapitel \ref{ssec:web} auf Seite
\pageref{ssec:web}.\\
Das gesamte Projekt besteht wie bereits h�ufiger erl�utert aus drei Teilen,
welche alle seperate installiert werden m�ssen. Zum Zeitpunkt des
Projektabschlusses steht kein Installationsprogramm zur Verf�gung. Daher wird
dem kooperativen Berufskolleg bei Auslieferung die vollst�ndige Applikation
installiert und eingerichtet. Die Installation soll in Zukunft vereinfacht
werden, damit diese ohne einen Entwickler durchgef�hrt werden kann.

