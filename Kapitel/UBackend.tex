Das Backend soll wie in Kapitel 2.3 beschrieben die Datenquelle f�r die
verschiedenen Clients sein. Umgesetzt werden soll dies �ber einen RESTful
Service. Es wird das Framework "`Spring"' eingesetzt. Dies erlaubt eine einfache
Erstellung von Web-Services und unterst�tzt den Programmierer mit Werkzeugen wie
Hibernate, die die Verwaltung der Datenbank stark vereinfacht und �nderungen an
der Modellierung direkt in der Datenbank umsetzt.
\subsection{Anforderungen}
Wie oben genannt soll das Backend als RESTful-Service gestaltet werden. D
\subsection{Implementierung}
\subsection{Tests }
\subsubsection{In Memory DB ?}
\subsection{Generischer Controller ??}
 