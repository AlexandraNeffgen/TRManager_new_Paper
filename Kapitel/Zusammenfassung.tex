St�rungen des Unterrichts durch Sch�ler zerren stets sowohl an den Nerven des
Lehrer, als auch an denen der Mitsch�ler. Eine ausreichende Besprechung und
Reflexion dieser St�rung w�hrend des Unterrichtes kostet allen beteiligten viel
Zeit, die dem Unterricht folglich fehlt. An dieser Stelle greift das
Traingsraumkozept, in dem es die Nachbereitung der St�rung mit dem Sch�ler aus
dem Unterricht in einen anderen betreuten Raum verlegt. F�r die Organisation
rund um den Traingsraum gibt es bisher keinerlei softwaregest�tzte  Hilfe,
daher beruht die bisherige Verwaltung auf Papier. \\
Im Rahmen der vorliegenden Studienarbeit wird eine Anwendung entwickelt, welche
die Lehrer unterst�tzen und vieles der Papier gest�tzten Verwaltung
ersetzt. Einleitend wird zun�chst die Idee und die Vorg�nge des Trainingsraum im
Detail erkl�rt. Folgen wird die konzeptionelle Entwicklung der Anwendung so wie
die technischen Grundlagen f�r das Verst�ndnis der Umsetzung. Die Anwendung
l�sst sich in mehrere Teile separieren, deren Umsetzung ebenfalls in verschieden
Kapitel gegliedert ist. \\
Das Fazit widmet sich den unterschiedlichen Funktionen und der Vollst�ndigkeit
ihrer Implementierung. Im Ausblick wird das weitere Potenzial der Anwendung
erfasst und die Bestrebungen der Autoren und Entwicklern beleuchtet.
W�hrend des gesamten Prozesse stand den Entwicklern das
Robert-Wetzlar-Berufskolleg Bonn zu Seite. Dieses lieferte zum einem
die Anforderungen und Testdaten und zum anderen wurde dort im laufenden Betrieb
die Anwendung getestet und evaluiert.  

