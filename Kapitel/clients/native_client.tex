Der native Client stellt die grafische Benutzeroberfl�che dar, die zus�tzlich
auch die Programm-Logik enth�lt. Der Client ist als \ac{MVC} Anwendung
konzipiert. Die einzelnen funktionalen Anforderungen an den Client sind im
Kapitel TODO aufgef�hrt. Im folgenden werden die einzelnen Aspekte und Elemente


\subsection{Design}

\subsubsection{Oberfl�che}
Das Hauptaugenmerk in der Gestaltung der Oberfl�che liegt wie in Kapitel
\ref{ssec:Usability} auf Seite \pageref{ssec:Usability} erkl�rt in der
Benutzerfreundlichkeit. Ebenfalls wichtig ist es aber auch ein ansprechendes und
modernes Design zu w�hlen, um die User zum arbeiten mit dem Programm zu
animieren. Aus diesen Gr�nden wurde sich f�r das Material Design von Google
entschieden. Dieses besticht durch sein minimalistisches Flat Design. Im Grunde
wird hier auf R�nder und Linien im klassischen Sinne verzichten. Zum hervorheben
von Interaktiven Schaltfl�chen werden diese mit einem Schatten versehen.
Nach dem Login �ber eine Seite welche sich auf das wesentliche beschr�nkt, wird
man auf eine Oberfl�che mit mehreren Reitern gef�hrt. F�r den Lehrer, welcher
Sch�ler im Traingsraum betreut (folgend User genannt) 


\subsection{Implementierung}
	F�r den nativen Client wird die Programmiersprache C\# so wie das Framework
	einer Windows Forms Anwendung verwendet. Wie in Kapitel \ref{Windows Forms}
	beschrieben verwendet eine solche Anwendung das .Net Framework. Dies erm�glicht 
	
	\subsubsection{Model}

\subsection{WEB-Request}

\subsection{Serialize JSON}

