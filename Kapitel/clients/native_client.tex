Der native Client stellt die grafische Benutzeroberfl�che dar, die zus�tzlich
auch die Programm-Logik enth�lt. Der Client ist als \adc{MVC} Anwendung
konzipiert. Die einzelnen funktionalen Anforderungen an den Client sind im
Kapitel TODO aufgef�hrt. Im folgenden werden die einzelnen Aspekte und Elemente
aufgef�hrt und erl�utert.
\subsection{Anforderungen}
Die Daten werden vom Backend �ber einen Restful-Client in JSON empfangen und dem
Benutzer in einer grafischen Oberfl�sche entsprechend pr�dentiert. Dabei werden
die Daten als Objekte in der Anwendung gehalten. Um ausfallsicher zu arbeiten
wird jede �nderung der Daten direkt an das Backend gesendet.

\subsection{Implementierung}
	F�r den nativen Client wird die Programmiersprache C\# so wie das Framework
	einer WindowsForms Anwendung verwendet. Wie in Kapitel \ref{Windows Forms}
	beschrieben verwendet eine solche Anwendung das .Net Framework. Dies erm�glicht 
	
	\subaubsection{Model}

\subsection{WEB-Request}

\subsection{Serialize JSON}

