Der native Client stellt die grafische Benutzeroberfl�che dar, die zus�tzlich
auch die Programm-Logik enth�lt. Der Client ist als \ac{MVC} Anwendung
konzipiert. Die einzelnen funktionalen Anforderungen an den Client sind im
Kapitel TODO aufgef�hrt. Im folgenden werden die einzelnen Aspekte und Elemente
aufgef�hrt und erl�utert.


\subsubsection{Design}

\paragraph{Oberfl�che}$~~$\\
Das Hauptaugenmerk in der Gestaltung der Oberfl�che liegt wie in Kapitel
\ref{ssec:Usability} auf Seite \pageref{ssec:Usability} erkl�rt in der
Benutzerfreundlichkeit. Ebenfalls wichtig ist es aber auch ein ansprechendes und
modernes Design zu w�hlen, um die User zum arbeiten mit dem Programm zu
animieren. Aus diesen Gr�nden wurde sich f�r das Material Design von Google
entschieden. Dieses besticht durch sein minimalistisches Flat Design. Im Grunde
wird hier auf R�nder und Linien im klassischen Sinne verzichten. Zum hervorheben
von Interaktiven Schaltfl�chen werden diese mit einem Schatten versehen.
Nach dem Login �ber eine Seite welche sich auf das wesentliche beschr�nkt, wird
man auf eine Oberfl�che mit mehreren Reitern gef�hrt. F�r den Lehrer, welcher
Sch�ler im Trainingsraum betreut (folgend User genannt) steht ein einzelner
Reiter zur Verf�gung. �ber diesen lassen sich alle Aufgaben des Users erledigen. 
Dem Administrator stehen noch weitere Reiter zur Verf�gung.
Unter dem Reiter "`Administration"' wird sowohl der Import, als auch der Export
der Datei gemanaged. Der Reiter "`Einstellungen"' erm�glicht es die
Serververbindung (IP, Port) anzugeben. Au�erdem kann hier die L�nge der
Unterrichtsstunden und der Pausen angegeben. Dies soll als Grundlage f�r sp�tere
Features dienen. 
Im Reiter "`Datens�tze verwalten"' hat der Administrator die M�glichkeit
zeitunabh�ngig die Informationen zu Incidents, Lehrern und Sch�lern zu
bearbeiten. 

\paragraph{Workflows}$~~$\\
Die einzelnen Workflows werden im folgenden erl�utert.

\setlength\parindent{0pt} \textbf{Sch�lerbesuch eintragen}

\textbf{Sch�ler aus dem Trainingsraum entlassen}

\textbf{Details zu einem Incident anzeigen}

\textbf{R�ckkehrer eintragen}  

\textbf{Import}

\textbf{Export}

\textbf{Servereinstellungen �ndern}

\textbf{Klasse �ndern}

\textbf{Klasse hinzuf�gen}

\textbf{Lehrer �ndern}

\textbf{Lehrer hinzuf�gen}    

\textbf{Incident �ndern}

\textbf{Incident hinzuf�gen}
