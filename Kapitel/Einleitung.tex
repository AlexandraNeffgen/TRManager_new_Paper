\subsection(Motivation)
In vielen Schulklassen s�mtlicher Schulformen kommt es immer wieder zu St�rungen des Unterrichtes durch Sch�ler. Dadurch verliert der Unterricht inhaltlich nicht nur an Qualit�t, sondern auch an Schwung. Um den entgegen zu wirken werden an vielen Schulen die Abwicklung der Unterrichtsst�rung und der Unterricht selber von einander getrennt. So werden die st�renden Sch�ler aus dem Unterricht heraus genommen und erhalten in einem so genannten Trainingsraum Unterst�tzung um die St�rung aufzuarbeiten und sich sozial weiterzuentwickeln. Dieses  Konzept wurde erstmals von Edward E. Ford in Phoenix , Arizona eingesetzt. Hier in Deutschland ist es daher sowohl unter dem Namen Trainingsraumprogramm, als auch Arizona-Modell bekannt. 
In Zusammenhang mit einem Besuch in dem Trainingsraum sind einige formale Dinge zu beachten. Gerade bei einem mehrfachen Besuch im Trainingsraum m�ssen neben dem Standardvorgehen weitere Prozesse angesto�en werden. Dies mit rein analogen Mitteln oder einfachen digitalen Ressourcen wie etwa Excel umzusetzen, ist nicht nur aufw�ndig, sondern auch anf�llig f�r Fehler und schwer zu �berblicken.
An dieser Stelle soll der Trainingsraum Manager Anwendung finden. Er soll nicht nur als Verwaltungstool dienen, sondern auch den betreuenden Lehrern als Orientierung f�r den Ablauf dienen. Dies soll obendrein die Einarbeitung neuer Lehrkr�fte in das Konzept erleichtern.

\subsection(These und Entwurf)
 
\subsection(Ziele und Aufgabenstellung))

\subsection(Stand der Technik)

\subsection(Aufbau der Arbeit)

