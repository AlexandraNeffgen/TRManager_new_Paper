\subsection{Motivation}
In vielen Schulklassen s�mtlicher Schulformen kommt es immer wieder zu St�rungen des Unterrichtes durch Sch�ler. 
Dadurch verliert der Unterricht inhaltlich nicht nur an Qualit�t, sondern auch an Schwung. 
Um den entgegen zu wirken werden an vielen Schulen die Abwicklung der Unterrichtsst�rung und der Unterricht 
selber von einander getrennt. So werden die st�renden Sch�ler aus dem Unterricht heraus genommen und erhalten 
in einem so genannten Trainingsraum Unterst�tzung um die St�rung aufzuarbeiten und sich sozial weiterzuentwickeln. 
Dieses  Konzept wurde erstmals von Edward E. Ford in Phoenix , Arizona eingesetzt. 
Hier in Deutschland ist es daher sowohl unter dem Namen Trainingsraumprogramm,
als auch Arizona-Modell bekannt. \\
In Zusammenhang mit einem Besuch in dem Trainingsraum sind einige formale Dinge zu beachten. 
Gerade bei einem mehrfachen Besuch im Trainingsraum m�ssen neben dem Standardvorgehen weitere Prozesse 
angesto�en werden. Dies mit rein analogen Mitteln oder einfachen digitalen Ressourcen wie etwa Excel umzusetzen, 
ist nicht nur aufw�ndig, sondern auch anf�llig f�r Fehler und schwer zu
�berblicken. \\
An dieser Stelle soll der Trainingsraum Manager Anwendung finden. 
Er soll nicht nur als Verwaltungstool dienen, sondern auch den betreuenden
Lehrern als Orientierung f�r den Ablauf dienen. Dies soll obendrein die
Einarbeitung neuer Lehrkr�fte in das Konzept erleichtern. \\

\subsection{These und Entwurf}
 Das Projekt TRManager soll es den P�dagogen erm�glichen sich ganz auf die
 F�rderung der Sch�ler zu konzentrieren. Durch eine grafische
 Benutzeroberfl�sche werden die P�dagogen durch das Prgramm gef�hrt. Sie
 erhalten eine �bersicht der m�glichen Aktionen, sowie eine �bersicht der
 anwesenden Sch�lern. Au�erdem werden die Regeln und Vorgaben wie etwa die
 mindest Anwesenheitsdauer eines Sch�lers implementiert. Dies erm�glicht eine
 einheitlichen Ablauf.
 S�mtliche Daten werden in einer Datenbank noch w�hrend der Laufzeit
 gespeichert. Zum einen l�sst sich dadurch die Datenkonsistent sicherstellen, so
 dass bei einem unerwartendem Beenden der Anwendung auch die aktuellen Daten in
 der Datenbank zu finden sind. Nach einem Neustart der Anwendung l�sst sich also
 direkt weiter arbeiten. Au�erdem ist so m�glich den aktuellen Stand des
 Trainingsraumes von mehreren Clients aus einzusehen.
 Grunds�tzlich soll eine Analyse der Daten mit Hilfe von Statistiken m�glich
 sein. Diese sollen nicht nur der Analyse der Sch�ler dienen, sondern auch der
 P�dagogen selbst.
 
\subsection{Ziele und Aufgabenstellung}
Ziel dieser Arbeit ist es, eine Server-Client-basierte Applikation zu erstellen.
Diese Applikation soll die Lehrer bei der Verwaltung des Trainingsraums
unterst�tzen, indem es Anwesenheiten der Sch�ler im Trainingsraum verfolgt und
Statistiken aufstellt. So l�sst sich der analoge Papierhaufen durch eine


\subsection{Stand der Technik}

\subsection{Aufbau der Arbeit}

