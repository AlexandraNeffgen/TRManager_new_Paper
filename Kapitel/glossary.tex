% \newglossaryentry{Android}{
%				name=Android,
%				description={Betriebssystem und Softwareplattform f�r mobile Ger�te, die
%				von der Open Handset Alliance(gegr�ndet von Google) entwickelt wird.
%				Basiert auf dem Linux-Kernel und wird quelloffen entwickelt.} 
%				}
%\newglossaryentry{Laufzettel}{
%				name=Laufzettel,
%				description={
%				}
%				}

\textbf{Android}\\
Betriebssystem und Softwareplattform f�r mobile Ger�te, die von der Open Handset
Alliance(gegr�ndet von Google) entwickelt wird. Basiert auf dem Linux-Kernel und
wird quelloffen entwickelt. \par 
\vspace{\baselineskip}
\textbf{Laufzettel}\\
Jeder Sch�ler der in den Trainingsraum geschickt wird erh�lt von seinem
schickenden Lehrer einen Laufzettel. Auf diesem steht die Art der
Unterrichtsst�rung, der Name des Sch�lers und seine Klasse drauf. Von dem
betreuenden Lehrer im Trainingsraum wird zus�tzlich die vom Programm generierte
Scheinnummer hinzugef�gt. Dieser Laufzettel stammt aus der analogen Verwaltung
des Trainingsraumes \par
\vspace{\baselineskip}
\textbf{idempotent}\\ 
Als idempotent bezeichnet man Arbeitsg�nge, die immer zu den gleichen
Ergebnissen f�hren, unabh�ngig davon, wie oft sie mit den gleichen Daten
wiederholt werden. Idempotente Arbeitsg�nge k�nnen zuf�llig oder absichtlich
wiederholt werden, ohne dass sie nachteilige Auswirkungen auf den Computer
haben. \par
\vspace{\baselineskip}