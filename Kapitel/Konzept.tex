\subsection{Trainingsraum}

\subsection{Design}

Die Applikation besteht im Grunde aus drei Teilen: Dem Frontend, dem Backend und
der Datenbank. Die Datenbank speichert die Daten, die erfasst werden und
pr�sentiert diese dem Backend. Das Backend verarbeitet die Daten, die es vom
Frontend und Datenbank bekommt, um es an beide weiterzureichen. Das Frontend
interpretiert die Daten, die es vom Backend geliefert bekommt und zeigt diese
dem Nutzer an und nimmt Eingaben vom Nutzer entgegen und schickt diese an das
Backend.\\

\subsection{Datenbank}

Zur Speicherung der Daten wurde ein \ac{ER}-Modell erstellt(REF_TO_PAGE). Dieses
beschreibt im ungef�hren den logischen Aufbau der Klassenstrukturen der Schule.
So hat jede Klasse einen Lehrer, jede Klasse mehrere Sch�ler und so weiter. 
%BILD ER-Modell


\subsection{Backend}

Das Backend dient als Schnittstelle zwischen Datenbank und Frontend und bereitet
die Daten die von Datenbank und Frontend kommen f�r den jeweils anderen auf.
Umgesetzt wird das Backend mit Spring. Die Zugriffe erfolgen �ber einen
\acs{REST}ful-Service, der laut REST-Spezifikation die normalen HTTP-Methoden
verwendet (GET, POST etc.). Die Verwendung von \ac{REST} erm�glicht es, die
Clients unabh�ngig vom Server zu entwerfen, da die Methoden standardisiert und
f�r jeden Client gleich sind. Auch andere Applikationen k�nnen so ohne Probleme
auf diesen Dienst zugreifen, um zum Beispiel die gesammelten Daten andersweitig
zu verwenden. 

\subsection{Frontends}

	\subsubsection{Native}

	\subsubsection{Web}
		Cordova (apache)
		Bootsraps
		http://v4-alpha.getbootstrap.com/examples/
	\subsubsection{Framework} 
		\paragraph{Zend}
	\subsubsection{Mobile}
	VPN leisten?
		\paragraph{IOS}
		\paragraph{Android}


