\subsection{Trainingsraum}
Das Konzept des Trainingsraumes beruht im wesentlichen auf den folgenden drei
Regeln:
\begin{itemize}
\item Jeder Sch�ler/jede Sch�lerinhat das Recht ungest�rt zu lernen.
\item Jeder Lehrer/jede Lehrerin hat das Recht ungest�rt zu unterrichten
\item Jeder muss die Rechte der Anderen respektieren 
\end{itemize}



\subsection{Design}

Die Applikation besteht im Grunde aus drei Teilen: Dem Frontend, dem Backend und
der Datenbank. Die Datenbank speichert die Daten, die erfasst werden und
pr�sentiert diese dem Backend. Das Backend verarbeitet die Daten, die es vom
Frontend und Datenbank bekommt, um es an beide weiterzureichen. Das Frontend
interpretiert die Daten, die es vom Backend geliefert bekommt und zeigt diese
dem Nutzer an und nimmt Eingaben vom Nutzer entgegen und schickt diese an das
Backend.\\

\subsection{Datenbank}



\subsection{Backend}

Das Backend dient als Schnittstelle zwischen Datenbank und Frontend und bereitet
die Daten die von Datenbank und Frontend kommen f�r den jeweils anderen auf.
Umgesetzt wird das Backend mit Spring. Die Zugriffe erfolgen �ber einen
\acs{REST}ful-Service, der laut REST-Spezifikation die normalen HTTP-Methoden
verwendet (GET, POST etc.). Die Verwendung von \ac{REST} erm�glicht es, die
Clients unabh�ngig vom Server zu entwerfen, da die Methoden standardisiert und
f�r jeden Client gleich sind. Auch andere Applikationen k�nnen so ohne Probleme
auf diesen Dienst zugreifen, um zum Beispiel die gesammelten Daten anderweitig
zu verwenden. 

\subsection{Clients}
	Die Clients haben den Zweck, dem Benutzer die gew�nschten Daten
	anzuzeigen.
	Der Zugriff auf diese Daten erfolgt generisch �ber
	\ac{HTTP}-Requests(\ac{REST}), so dass f�r jeden Client die gleiche Technik
	verwendet werden kann. Geplant sind mehrere Clients:
	\subsubsection{Native}
		Der native Client verwendet die Bibliotheken von Windows Forms und soll in C\#
		programmiert werden. Dem Benutzer soll auf einem Blick angezeigt werden, wer
		gerade im Trainingsraum ist, wie lange dieser schon darin ist und ob er den
		Raum verlassen darf. Benutzern mit administrativen Rechten soll auch eine
		Verwaltung der Datens�tze erm�glich werden. Diese Funktionen werden f�r
		"`normale"' Benutzer ausgeblendet.
		
	\subsubsection{Web}
		Der Web-Client soll �hnlich dem nativen Client dem Benutzer die derzeitig
		anwesenden Sch�ler anzeigen, jedoch wird dieser als Webseite bereitgestellt.
		Die einzusetzenden Techniken werden evaluiert 
	\subsubsection{Framework} 
		\paragraph{Zend(Weg?)}
	\subsubsection{Mobile}
	VPN leisten?
		\paragraph{IOS}
		\paragraph{Android}


