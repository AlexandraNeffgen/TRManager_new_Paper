BLAVAALBLALBLVLS
einleitender Absatz
!!!

\subsection{Technische Komponenten}

\subsubsection{Web-Applikation}

\subsubsection{Mobile Applikation}

\subsubsection{Statistik-Tools}

\subsubsection{Backend-Erweiterungen}


\subsection{Wirtschaftlicher Faktor}
Das Projekt hat als Endergebnis ein solide funktionierendes Programm, welches
auf dem aktuellen Markt keine Konkurrenz hat. Zudem wird das Konzept des
Trainingsraumes immer mehr auch durch die Schulministerien der L�nder
propagiert. Eine Vermarktung dieser Anwendung scheint daher der n�chst logische
Schritt. Bevor dieser Schritt angegangen werden kann, sollte das Programm
zun�chst durch ein dokumentiertes Pilotverfahren gebracht werden. Hier steht den
ENtwicklern das Robert-Wetzlar-Berufskolleg zur Seite. Dieses verwendet bereits
fr�here Versionen des Programmes und hat das Trainingsraum-Konzet in den
Schulaltag bereits vollst�ndig integriert.\\
Um das Programm zu vermarkten muss nat�rlich zum einen Marketing stattfinden und
zum anderen die Lizenzierung und damit auch Kosten

 Programm an die Schulen bringen


\subsubsection{Marketing}


\subsubsection{Lizensierung}
Sobald man sich mit der Vermarktung eines Produktes besch�ftigt, kommt
zwangsl�ufig die Frage nach der Lizenzierung. Bevor man sich entscheiden kann
oder auch nur mit den unterschiedlichen Lizenzen auseinander setzen kann sollten
die eigenen Anforderungen klar definiert sein. Die Entwickler dieses Projektes
haben dabei die folgenden Rahmenbedingungen herausgearbeitet
\begin{itemize}
  \item Quelloffen
  \item Namensnennung
  \item keine Manipulation des Codes 
  \item keine (kommerzielle) Wiederverwendung des Codes
  \item kein Vertrieb der Software durch Dritte
\end{itemize}
Die Rahmenbedingungen m�ssen alle evaluiert und auf ihre Tauglichkeit gepr�ft
werden. Des Weiteren ist unter den vielen unterschiedlichen m�glichen
Lizenzmodellen und Lizenzarten das passende zu w�hlen und ggf Abstriche zu
machen.


