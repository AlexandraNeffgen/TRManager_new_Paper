Im Rahmen dieser Arbeit konnten bereits viele Features umgesetzt werden. Dennoch
stehen noch einige Features aus. Diese sind dabei keine Grundfunktionen, sondern
Funktionen, die immer wieder von den Nutzern gew�nscht wurden, oder aber die
Entwickler f�r sinnvoll erachten. 
Au�erdem erm�glicht das Produkt eine wirtschaftliche Betrachtung, die in Kapitel
\ref{ssec:wFaktor} auf Seite \pageref{ssec:wFaktor} genauer erl�utert wird.

\subsection{Technische Komponenten}
Auf technischer Seite gibt es einige Aspekte die im Anschluss an das Projekt
noch weiter ausgearbeitet werden. Die Hauptpunkte werden in den folgenden
Abschnitten vorgestellt.

\subsubsection{Web-Applikation}
Der Webclient hat bisher nur einen sehr geringen Funktionsumfang. Hier ist es
gerade mal m�glich die Hauptaufgaben des eines betreuenden Lehrer zu bew�ltigen.
In Zukunft soll dieser Client einen �hnlichen Funktionsumfang wie der native
Client aufweisen. Dazu sollten allerdings weitere Sicherheitsaspekte betrachtet
werden.

\subsubsection{Mobile Applikation}
Zu Beginn der Arbeit wurde bereits �ber eine mobile Applikations nachgedacht.
Durch die Web-Applikation r�ckte diese allerdings in den Hintergrund.
nichtsdestotrotz stellt eine App f�r Tablets und Smartphones ein attraktives
Angebot f�r moderne Schulen dar. Hier besteht demnach Potenzial f�r weitere
Entwicklungen.

\subsubsection{Statistik-Tools}
F�r die Auswertung der Daten bietet sich ein eigenes Statistik-Tool an. Eine
solche Anforderung taucht auch schon in den User-Stories auf. Zwar ist �ber den
Reiter Datensatzverwaltung eine Liste aller Eintr�ge abrufbar, aber als
Statistik geht diese wohl nicht durch. Hier ist eine separate Anwendung oder
entsprechende Erweiterungen anzugehen.

\subsubsection{Backend-Erweiterungen}
Verschl�sselung 


\subsection{Wirtschaftlicher Faktor}\label{ssec:wFaktor}
Das Projekt hat als Endergebnis ein solide funktionierendes Programm, welches
auf dem aktuellen Markt keine Konkurrenz hat. Zudem wird das Konzept des
Trainingsraumes immer mehr auch durch die Schulministerien der L�nder
propagiert. Eine Vermarktung dieser Anwendung scheint daher der n�chst logische
Schritt. Bevor dieser Schritt angegangen werden kann, sollte das Programm
zun�chst durch ein dokumentiertes Pilotverfahren gebracht werden. Hier steht den
Entwicklern das Robert-Wetzlar-Berufskolleg zur Seite. Dieses verwendet bereits
fr�here Versionen des Programmes und hat das Trainingsraum-Konzept in den
Schulaltag bereits vollst�ndig integriert.\\
Um das Programm zu vermarkten muss nat�rlich zum einen Marketing stattfinden und
zum anderen die Lizenzierung und damit auch Kosten f�r die Kunden.

\subsubsection{Marketing}
Das Trainingskonzept wird nicht nur von den Schulministerien der L�nder beworben
sondern auch von unterschiedlichen Sozialp�dagogen. So zum Beispiel auch Dr.
Heidrun Br�ndel, Dipl.-Psych., und Erika Simon, Lehrerin. Zusamen haben die
beiden das Buch "`Die Trainingsraum-Methode"' herausgebracht. In diesem beschreiben sie unter anderem in
einem Kapitel die B�rokratie der Methode 
\footnote{\bibentry{2013Trainingsraum-Methode}}. 
Daraus bieten sich den Entwicklern dieser Anwendung nun zwei Wege des Marketing.
F�r ein gro�fl�chiges Marketing k�nnte mit den beiden Parteien (Buchautoren und
Schuliministerium) Kontakt aufgenommen werden um eine Nennung des Programmes auf
deren Webseiten anzustreben. Dazu sollten aber neben dem Anwenderhandbuch
(siehe Anhang \ref{ssec:Anwenderhandbuch}) auch entsprechendes
Pr�sentationsmaterial und mindestens eine Testumgebung des Programmes zur
Verf�gung stehen sollte. Eine solche Testumgebung steht bereits zur Verf�gung
und ist �ber entsprechende Verbindungsdaten und Authentifizierungsdaten
erreichbar.
Neben diesem Weg ist nat�rlich auch das direkte Anschreiben von Schulen eine
M�glichkeit und sollte zuk�nftig in Betracht gezogen werden.
der pers�nliche Kontakt zu Lehrern und Dozenten von Lehrerschulungen sollte den
Prozess der Verbreitung ebenfalls positiv beeinflussen.

\subsubsection{Lizensierung}
Sobald man sich mit der Vermarktung eines Produktes besch�ftigt, kommt
zwangsl�ufig die Frage nach der Lizenzierung. Bevor man sich entscheiden kann
oder auch nur mit den unterschiedlichen Lizenzen auseinander setzen kann sollten
die eigenen Anforderungen klar definiert sein. Die Entwickler dieses Projektes
haben dabei die folgenden Rahmenbedingungen herausgearbeitet
\begin{itemize}
  \item Quelloffen
  \item Namensnennung
  \item keine Manipulation des Codes 
  \item keine (kommerzielle) Wiederverwendung des Codes
  \item kein Vertrieb der Software durch Dritte
\end{itemize}
Die Rahmenbedingungen m�ssen alle evaluiert und auf ihre Tauglichkeit gepr�ft
werden. Des Weiteren ist unter den vielen unterschiedlichen m�glichen
Lizenzmodellen und Lizenzarten das passende zu w�hlen und ggf Abstriche zu
machen.


